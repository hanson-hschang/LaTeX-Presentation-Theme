\section{Format} 

\begin{frame}
  \frametitle{Metadata Settings}
  \framesubtitle{and beamer class settings}
  
  \paragraph{BEAMER class}
  The class is default to \Verb|presentation| mode. 
  For printing handouts, use \Verb|\documentclass[handout]{beamer}| instead. 
  \alert{Handout mode Incomplete yet.}

  \paragraph{Metadata fields}
  Several metadata fields can be set in the preamble:
  \begin{itemize}
    \item \Verb|\title[short title]{full title}|
    \item \Verb|\subtitle{subtitle}|
    \item \Verb|\author[short name]{full name}|
    \item \Verb|\institute[short name]{full name}|
    \item \Verb|\date{date}|
    \item \Verb|\logo{graphic}|
    \item \Verb|\titlegraphic{graphic}| (optional)
  \end{itemize}
\end{frame}

\begin{frame}
  \frametitle{Frame Settings}

  \paragraph{Frame options}
  A \Verb|frame| environment can take several options:
  \begin{itemize}
    \item \Verb|plain|: removes header and footer.
    \item \Verb|noframenumbering|: excludes the frame from slide numbering.
    \item \Verb|label=label_name|: assigns a label to the frame for hyperlinking.
    \item \Verb|t|, \Verb|c|, \Verb|b|: aligns the content at the top, center, or bottom of the frame.
  \end{itemize}
  
  \paragraph{No frame number indexing}
  To remove frame number indexing at both the navigation bar and the footer, wrap the \Verb|frame| environment with the \Verb|noframenumber| environment defined in the theme.

  \paragraph{Frame title and subtitle}
  Use the \Verb|\frametitle{title}| and \Verb|\framesubtitle{subtitle}| commands within a \Verb|frame| environment to set the title and subtitle of the frame, respectively.

\end{frame}

\begin{frame}
  \frametitle{Title Page \& TOC Slides}

  \paragraph{Title Page}
  The title page is created using the \Verb|\titlepage| command within a \Verb|frame| environment.
  It is recommended to use the \Verb|plain| and \Verb|noframenumbering| options for the title page frame.

  \paragraph{Table of Contents Slide}
  A table of contents (TOC) slide is created using the \Verb|\tableofcontents| command within a \Verb|frame| environment.
  It is also recommended to use the \Verb|noframenumbering| option for the TOC slide.
  Note that a TOC slide highlighting the current section is automatically added at the beginning of each section.

\end{frame}

% TODO: If you need a button, label the frames accordingly then using \hyperlink{label_of_the_dest_frame}{\beamerbutton{Name of the dest. frame}
\begin{frame}[label=objectives]{Objectives 
  \hyperlink{scope}{\beamerbutton{Scope}}
}
	\begin{block}{Sample Block Title}
		This block presents a \alert{key concept} that is crucial for understanding the topic.
	\end{block}
	\begin{alertblock}{Sample Alert Block Title}
		This block presents a more alarming \alert{key concept} that is crucial for understanding the topic.
	\end{alertblock}
	%% TODO: You can add the note here
	\note{}
\end{frame}

\begin{frame}{Actors \& Features}
	\textbf{Actors:}
									
	\textbf{Features:}
	%% TODO: You can add the note here
	\note{}
\end{frame}


\begin{frame}{Contributions}				
	\begin{block}{Scientific Contribution}
	\end{block}						
	\begin{block}{Real-world Contribution}
	\end{block}					
	%% TODO: You can add the note here
	\note{}
\end{frame}